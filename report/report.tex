\documentclass{article}

\usepackage{hyperref}
\usepackage{xcolor}
\usepackage{graphicx}
\usepackage{array} % fixed with table cells

\title{Biodivne Boolean Models: A Comprehensive Logical Modelling Benchmark}
\author{Samuel Pastva}
\date{}

% Fixed width table columns
\newcolumntype{L}[1]{>{\raggedright\let\newline\\\arraybackslash\hspace{0pt}}m{#1}}
\newcolumntype{C}[1]{>{\centering\let\newline\\\arraybackslash\hspace{0pt}}m{#1}}
\newcolumntype{R}[1]{>{\raggedleft\let\newline\\\arraybackslash\hspace{0pt}}m{#1}}

\begin{document}
\maketitle

\begin{abstract}
	The space of powerful tools for computational analysis of logical models (represented as Boolean networks) has been growing considerably in the recent years. However, comparing the validity, efficiency, and scalability of such tools necessitates a comprehensive benchmark set of realistic Boolean networks.
	
	At the moment, this need is largely served by databases of biological models such as CellCollective, GINsim, or Biomodels. However, these databases are more focused on human curated, biologically relevant networks, and are therefore limited in scope. Furthermore, the models in such databases may not always be available as a single dataset or in the desired format. This requires additional time for preprocessing and data collection, which the tool authors are often unwilling or unable to spend.
	
	In this technical report, we describe a comprehensive, open-source benchmark dataset that has been created by surveying the aforementioned databases, as well as a large body of other literature to obtain as many realistic Boolean networks as possible. To make the dataset useful to a wide range of tool maintainers, we provide the models in different machine-readable formats and ensure all models are valid and logically consistent using an automated static-analysis workflow. At the moment of writing, the dataset comprises more than 210 networks.
\end{abstract}

\section{Introduction}

Logical models provide a very useful and simple framework for describing complex biological phenomena. Likely the most common mechanism for formalising executable logical models are Boolean networks~\cite{bn-intro}. In recent years, we have seen a rapid development of new tools and algorithms for analysis of large Boolean networks. However, in many instances, it is hard to assess usefulness and scalability of such tools due to a lack of commonly recognised ``benchmark dataset'' of networks on which the tools can be compared.

This purpose is often served by models obtained from databases maintained by the authors of some of the larger modelling tools, such as CellCollective~\cite{cell-collective}, GINsim~\cite{ginsim}, or Biomodels~\cite{biomodels}. However, these models are often hard to obtain in bulk or may require additional processing (e.g. to convert into an appropriate format). Additionally, publication authors often modify the models in minor ways (e.g. by tweaking valuations of network inputs), which prevents meaningful comparisons between publications. Finally, these databases are far from comprehensive, so a wide range of models is often omitted.

As a result, most papers develop an ad hoc benchmark set that is often partially proprietary and hard or impossible to replicate and compare to. In this technical report, we describe a comprehensive, open-source benchmark dataset that can be used for this purpose instead.

\textbf{Disclaimer: The model dataset as well as this report will be gradually updated as new models and data become available. If possible, please refer to the latest version of this report on arXiv and the latest revision of the dataset in the Github repository.}

\section{Goals and scope}

\paragraph{Eligible models} To cover the widest possible assortment of realistic models, we currently allow submission of any logical model which can be reliably retrieved from a known database, repository, or an associated publication. The only requirement is that the model must be based on a real biological system. That is, we do not accept randomly generated models. However, we do not require any specific level of curation either---the model can be hand made, automatically synthesised, or anything in between. Similarly, we accept many different model formats and we are willing to assist with translation if a machine-friendly format is not available. We also accept multi-valued models, however, for now the dataset only contains their Booleanized derivatives.

\paragraph{Editions} The goal is to gradually evolve the dataset by adding new models (or by removing problematic/redundant models). Consequently, we plan to regularly release \emph{editions} of the dataset, such that the users can refer to the whole history of editions as necessary. Preliminary plan is to release editions in roughly yearly intervals. Initially, editions will be available as Github releases, with a possible future addition of Zenodo artefacts. In each edition, the user can then pick models with specific properties (size, keywords, source, etc.).

\paragraph{Model metadata} While we do not perform any biological curation of models, we collect basic meta-data about each model to make filtering and search easier. In particular, each model has:
\begin{itemize}
	\item A \emph{numeric identifier} that is unique within a specific dataset edition.
	\item A human-readable name. For simplicity, the name is limited to numbers, capital letters and the dash symbol (e.g. \texttt{MODEL-NAME-5}). To improve legibility, we may use spaces instead of dashes in text that is not meant to be machine readable (i.e. \texttt{MODEL NAME 5}).
	\item The DOI of the \emph{associated publication} and its \emph{bibliographic entry} (in Bibtex). Note that a single publication can contain multiple models---some DOIs thus appear in relation to multiple models.
	\item The URL where the model data was downloaded. This can be a list of URLs if the model is available from multiple sources. This can also be the publication DOI if the model is available directly through the published supplementary data.
	\item Basic structural metadata, such as the number of model \emph{variables}, \emph{inputs}, and \emph{regulations}. The plan is to also incorporate additional structural measures of the regulatory graph later (e.g. feedback-vertex-set, SCC sizes, etc.), once additional static analysis steps are added.
	\item A set of curated \emph{keywords}. Generally, these represent additional technical metadata, such as listing the databases where the model is available, or whether the model is based on multi-valued logic. At the moment, the dataset does not contain any biological keywords (e.g. cancer, differentiation, etc.). However, we are open to incorporating any community suggestions for additional keywords.
	\item A markdown document with any additional notes or relevant information about the model.
\end{itemize}

For each model, the metadata is summarised in a human-readable markdown \texttt{README.md} file, as well as a machine-readable \texttt{.json} document.


\paragraph{Contributions}

The dataset is managed using a versioned Github repository.\footnote{\url{https://github.com/sybila/biodivine-boolean-models/}} Submitting a new model into the collection can be thus performed by anyone using the pull-request functionality (project readme has more detailed instructions about preparing such pull requests). Alternatively, we also accept submissions using issues: Simply create an issue with the bibliographic entry of the model you wish to include and we will try to incorporate it as soon as possible.

\section{Technical information}
\label{section:technical}

The dataset is organized as follows:

\begin{itemize}
	\item \texttt{/models} contains the whole dataset with all model and metadata files.
	\item \texttt{/sources} directory contains the original machine-readable source files that are used to generate the \texttt{models} directory.
	\item \texttt{/report} directory contains the LaTeX source files for this report.
	\item \texttt{/sync.py} is a Python script for model processing and static analysis (takes models from \texttt{/sources} and generates files in \texttt{/models}).
	\item \texttt{/bundle.py} is a Python script for creating model bundle archives. These can include model variants with different input representation, or a subset of the collection filtered according to some basic conditions.
\end{itemize}

For more information on how to use \texttt{sync.py} and \texttt{bundle.py} to work with the dataset, see the project readme file.

\paragraph{Keywords} The metadata for each model can contain a set of keywords. In theory, any keyword can be added to identify specific classes of models. However, at the moment, we specifically recognise the following keywords:

\begin{itemize}
	\item \texttt{curated}: A curated model is not only based on biological information, but is also (partially) validated by a human. That is, a model resulting from a purely automatic translation or inference can be included in the dataset, but is not automatically considered curated.
	\item \texttt{repaired}: A model where one or more logical inconsistencies have been resolved (a complete list of possible repairs is given later in this report). The rationale behind each modification should be listed in model notes.
	\item \texttt{multi-valued}: A model whose original version is multi-valued, but which is presented in a Booleanized form (the original multi-valued model should be included in the \texttt{sources} directory).
	\item \texttt{cell-collective}/\texttt{ginsim}/\texttt{biomodels}/\texttt{covid-disease-map}: Indicates that the model is available in the corresponding database (or multiple databases).
	\item \texttt{casq}: Indicates that the model has been translated from a CellDesigner pathway using the CaSQ tool~\cite{casq}.
\end{itemize}

\paragraph{Formats and representation} At the moment, we support three major model formats for BN representation: 

\begin{itemize}
	\item \texttt{.bnet}; a simple plaintext file with the logical update functions~\cite{pyboolnet}.
	\item \texttt{.aeon}; also a plaintext format, but one that also incorporates the regulatory network of the model (with monotonicity and essentiality)~\cite{aeon}.
	\item \texttt{.sbml}; XML-based format with support for additional meta-data (model layout, entity references and comments, etc.)~\cite{sbml-qual}.
\end{itemize}

Each model is available in all three formats. Any of the three formats can be also used as a model source (however, only limited static analysis is available for \texttt{.bnet} models due to the missing regulatory network). To ensure compatibility across all tools, we also perform model variable renaming when necessary. For example, \texttt{.sbml} supports (almost) arbitrary strings as variable names, while other formats follow a more typical C-like naming requirements. As such, regardless of the source format, variable names are normalized so that they are valid in any format.

\paragraph{Model inputs} We also normalize the representation of model \emph{inputs} and \emph{constants}. Formally, an input or a constant is an entity with no incoming regulations, and either an identity (input) or \texttt{true}/\texttt{false} (constant) update function. The update function for inputs can be also missing entirely to indicate that the value is unknown/undetermined.

The specific representation of such variables is typically at the discretion of the model authors and differs from model to model. For example, some authors use constant update functions for inputs with a ``typical'' or ``assumed'' value, while other authors use the identity function, and other authors omit the functions entirely, only describing the expected values in text.

To make the resulting models as canonical as possible, we opted to \emph{omit the update function} for every constant or input variable. If you wish to use the original setting as published by the authors, this information is still available in the \texttt{/sources} directory, or as part of the original publication. Furthermore, using the \texttt{bundle.py} script, you can generate variants of the dataset with a different input setting (constant true/constant false/identity function/free).

For benchmarking and tool comparison, we generally recommend sampling a non-trivial subspace of input valuations, as the behaviour of the model can often differ substantially based on the values of inputs. The model with all input values erased is therefore a good starting point for such sampling.

\paragraph{Multi-valued networks} Some logical models are not Boolean, but multi-valued. Our dataset also includes models of this type. However, every such model is only given in the Booleanized format where the multi-valued entities are expanded into multiple Boolean variables. At the moment, the Booleanization is performed manually using the GINsim~\cite{ginsim} software. In the future, we may include this as an automated step in the analysis workflow. Every Booleanized multi-valued model is marked with the \texttt{multi-valued} keyword. The original multi-valued model is also available in the corresponding \texttt{sources} sub-directory.

\paragraph{Duplicates} Some models are available from multiple sources (typically multiple model databases). To eliminate duplicates, we typically only include a single version of the same model, but list multiple model URLs as possible sources (note that different databases can sometimes have slightly different versions of the same model). The information about duplicates and model-database mapping is also tracked in a shared spreadsheet.\footnote{\url{https://docs.google.com/spreadsheets/d/1YFMaT3SKOwsReW3r6umv7Atqjoe1AZJmeMQ3HBPnIUI/edit?usp=sharing}} Note that some publications can still contain multiple models, in which case we include all such models (each model is linked to the single shared publication). Similarly, multiple publications can describe variations or derivatives of a single model. We then typically include the variations as separate models, as long as there is a non-trivial difference from the original model (e.g. new/removed variables or regulations).

\paragraph{Model repair} Many published models contain logical \emph{inconsistencies} that we are able to detect as part of our static analysis workflow. Such issues have to be resolved before the model can be included in the dataset. For every model where such ``repair'' operation was performed, we use the \texttt{repaired} keyword and describe the repair process in the model notes. Furthermore, the original (unaltered) model file should be available in the \texttt{sources} directory. In general, we recognise the following types of problems:

\begin{itemize}
	\item Monotonicity: When a model declares a regulation as (positively or negatively) monotonous, but the actual update function does not adhere to this assumption, we remove the monotonicity requirement from the regulation.
	\item Essentiality: When a model declares a regulation but the regulator has no impact on the output of the update function, we mark the regulation as non-essential. This problem has two variants: Either the regulator is completely unused (in which case this is likely the intention of the model author), or the regulator appears in the update function, but has no actual impact on its output (in which case it is likely that the model author was not aware of this problem).
	\item Redundant variables: We expect the dependency graph of each model to be weakly connected, or at least to not contain completely disconnected variables. If such variables are found, they are removed.
	\item Typos and other issues: Sometimes, the logical inconsistency is due to an obvious typo in the model specification (e.g. substitution of two similar variable names), in which case we fix the typo to the best of our ability.
\end{itemize}

The goal of each repair is to retain the original update functions of the model (as these are typically crucial for reproducibility). We thus typically adjust the regulatory network to match the actual update functions.

\section{Models}

In the rest of the report, we give a complete enumeration of all models included in the dataset with basic metadata and associated bibliographic entries. A machine readable (\texttt{.csv}) version of this summary is available as part of each dataset edition.

\begin{center}
	\begin{tabular}{ | C{16pt} | C{144pt} | C{25pt} | C{25pt} | C{25pt} | C{36pt} | }
		\hline
		ID & Name & Vars. & Inps. & Regs. & Source \\ \hline
		001 & \texttt{SIGNALING IN MACROPHAGE ACTIVATION} & 302 & 19 & 533 & \cite{bbm-001} \\ \hline
		002 & \texttt{SIGNAL TRANSDUCTION IN~FIBROBLASTS} & 130 & 9 & 557 & \cite{bbm-002} \\ \hline
		003 & \texttt{MAMMALIAN CELL CYCLE} & 19 & 1 & 51 & \cite{bbm-003} \\ \hline
		004 & \texttt{ERBB RECEPTOR SIGNALING} & 225 & 22 & 1100 & \cite{bbm-004}  \\ \hline
		005 & \texttt{FA/BRCA PATHWAY} & 28 & 0 & 123 & \cite{bbm-005} \\ \hline
		006 & \texttt{HGF SIGNALING IN~KERATINOCYTES} & 62 & 6 & 103 & \cite{bbm-006} \\ \hline
		007 & \texttt{CORTICAL AREA DEVELOPMENT} & 5 & 0 & 14 & \cite{bbm-007} \\ \hline
		008 & \texttt{DEATH RECEPTOR SIGNALING} & 25 & 3 & 45 & \cite{bbm-008-150} \\ \hline
		009 & \texttt{YEAST APOPTOSIS} & 60 & 13 & 114 & \cite{bbm-009} \\ \hline
		010 & \texttt{CARDIAC-DEVELOPMENT} & 13 & 2 & 37 & \cite{bbm-010} \\ \hline
		011 & \texttt{GUARD CELL ABSCISIC ACID~SIGNALING} & 40 & 4 & 78 & \cite{bbm-011} \\ \hline		
		012 & \texttt{T-CELL RECEPTOR SIGNALING} & 94 & 7 & 158 & \cite{bbm-012} \\ \hline
		013 & \texttt{CHOLESTEROL REGULATORY~PATHWAY} & 32 & 2 & 41 & \cite{bbm-013} \\ \hline
		014 & \texttt{T-LGL SURVIVAL NETWORK 2008} & 54 & 7 & 193 & \cite{bbm-014} \\ \hline
		015 & \texttt{NEUROTRANSMITTER SIGNALING~PATHWAY} & 14 & 2 & 20 & \cite{bbm-015} \\ \hline
		016 & \texttt{IL-1 SIGNALING} & 104 & 14 & 218 & \cite{bbm-016-019} \\ \hline
		017 & \texttt{DIFFERENTIATION OF T-LYMPHOCYTES}  & 41 & 9 & 97 & \cite{bbm-017} \\ \hline
		018 & \texttt{EGFR-ERBB SIGNALING} & 76 & 28 & 226 & \cite{bbm-018} \\ \hline
		019 & \texttt{IL-6 SIGNALING} & 71 & 15 & 149 & \cite{bbm-016-019} \\ \hline
		020 & \texttt{APOPTOSIS NETWORK} & 39 & 2 & 73 & \cite{bbm-020} \\ \hline
		021 & \texttt{BODY SEGMENTATION IN DROSOPHILA 2013} & 14 & 3 & 29 & \cite{bbm-021} \\ \hline
		022 & \texttt{B-CELL DIFFERENTIATION} & 17 & 5 & 39 & \cite{bbm-022} \\ \hline
		023 & \texttt{MAMMALIAN CELL CYCLE 2006} & 9 & 1 & 34 & \cite{bbm-023} \\ \hline
		024 & \texttt{BUDDING YEAST CELL CYCLE} & 16 & 4 & 42 & \cite{bbm-024} \\ \hline
		025 & \texttt{T-LGL SURVIVAL NETWORK 2011} & 54 & 6 & 195 & \cite{bbm-025-074} \\ \hline
		026 & \texttt{BUDDING YEAST CELL~CYCLE~2009} & 18 & 0 & 59 & \cite{bbm-026} \\ \hline
		027 & \texttt{WG PATHWAY OF DROSOPHILA} & 12 & 14 & 29 & \cite{bbm-drosophila} \\ \hline
		028 & \texttt{VEGF PATHWAY OF DROSOPHILA} & 10 & 8 & 18 & \cite{bbm-drosophila} \\ \hline
		029 & \texttt{TOLL PATHWAY OF DROSOPHILA} & 9 & 2 & 11 & \cite{bbm-drosophila} \\ \hline
		030 & \texttt{SPZ NETWORK OF DROSOPHILA} & 18 & 6 & 28 & \cite{bbm-drosophila} \\ \hline
	\end{tabular}	

	\begin{tabular}{ | C{16pt} | C{149pt} | C{25pt} | C{25pt} | C{25pt} | C{36pt} | }
		\hline
		ID & Name & Vars. & Inps. & Regs. & Source \\ 
		\hline
		031 & \texttt{CELL CYCLE TRANSCRIPTION} & 9 & 0 & 19 & \cite{bbm-031} \\ 
		\hline
		032 & \texttt{T-CELL SIGNALLING 2006} & 37 & 3 & 53 & \cite{bbm-032} \\ 
		\hline
		033 & \texttt{BT474 BREAST CELL LINE~LONG~TERM} & 19 & 5 & 68 & \cite{bbm-breast-cell-line} \\ 
		\hline
		034 & \texttt{HCC1954 BREAST CELL LINE~LONG~TERM} & 19 & 4 & 68 & \cite{bbm-breast-cell-line} \\ 
		\hline
		035 & \texttt{BT474 BREAST CELL LINE~SHORT~TERM} & 11 & 5 & 46 & \cite{bbm-breast-cell-line} \\ 
		\hline
		036 & \texttt{HCC1954 BREAST CELL LINE~SHORT~TERM} & 11 & 5 & 46 & \cite{bbm-breast-cell-line} \\ 
		\hline
		037 & \texttt{SKBR3 BREAST CELL LINE~SHORT~TERM} & 11 & 5 & 41 & \cite{bbm-breast-cell-line} \\ 
		\hline
		038 & \texttt{SKBR3 BREAST CELL LINE~LONG~TERM} & 21 & 4 & 81 & \cite{bbm-breast-cell-line} \\ 
		\hline
		039 & \texttt{HIV-1 INTERACTIONS WITH T-CELL SIGNALING} & 124 & 14 & 368 & \cite{bbm-039} \\ 
		\hline
		040 & \texttt{T-CELL DIFFERENTIATION} & 19 & 4 & 34 & \cite{bbm-040} \\ 
		\hline
		041 & \texttt{INFLUENZA VIRUS REPLICATION~CYCLE} & 120 & 11 & 302 & \cite{bbm-041} \\ 
		\hline
		042 & \texttt{TOL REGULATORY NETWORK} & 14 & 10 & 48 & \cite{bbm-042} \\ 
		\hline
		043 & \texttt{BORDETELLA BRONCHISEPTICA} & 33 & 0 & 79 & \cite{bbm-043-044-046} \\ 
		\hline
		044 & \texttt{TRICHOSTRONGYLUS RETORTAEFORMIS} & 25 & 1 & 58 & \cite{bbm-043-044-046} \\ 
		\hline
		045 & \texttt{HH PATHWAY OF DROSOPHILA} & 11 & 13 & 32 & \cite{bbm-drosophila} \\ 
		\hline
		046 & \texttt{B~BRONCHISEPTICA AND T~RETORTAEFORMIS} & 52 & 1 & 135 & \cite{bbm-043-044-046} \\ 
		\hline
		047 & \texttt{FGF PATHWAY OF DROSOPHILA} & 14 & 9 & 24 & \cite{bbm-drosophila} \\ 
		\hline
		048 & \texttt{GLUCOSE REPRESSION SIGNALING~2009} & 55 & 18 & 97 & \cite{bbm-048-173} \\
		\hline
		049 & \texttt{OXIDATIVE STRESS PATHWAY} & 18 & 1 & 32 & \cite{bbm-049} \\
		\hline
		050 & \texttt{CD4 T-CELL SIGNALING} & 154 & 34 & 351 & \cite{bbm-050} \\
		\hline
		051 & \texttt{COLITIS ASSOCIATED COLON~CANCER} & 69 & 1 & 153 & \cite{bbm-051} \\
		\hline
		052 & \texttt{SEPTATION INITIATION NETWORK} & 23 & 8 & 50 & \cite{bbm-052} \\
		\hline
		053 & \texttt{PREDICTING VARIABILITIES IN~CARDIAC GENE} & 13 & 2 & 37 & \cite{bbm-053} \\ 
		\hline
		054 & \texttt{PC12 CELL DIFFERENTIATION} & 61 & 1 & 108 & \cite{bbm-054} \\
		\hline
		055 & \texttt{HUMAN GONADAL SEX DETERMINATION} & 19 & 0 & 79 & \cite{bbm-055} \\
		\hline
		056 & \texttt{IGVH MUTATIONS IN~LYMPHOCYTIC LEUKEMIA} & 66 & 25 & 125 & \cite{bbm-056} \\
		\hline
	\end{tabular}

	\begin{tabular}{ | C{16pt} | C{149pt} | C{25pt} | C{25pt} | C{25pt} | C{36pt} | }
		\hline
		ID & Name & Vars. & Inps. & Regs. & Source \\ 
		\hline
		057 & \texttt{FANCONI ANEMIA AND CHECKPOINT RECOVERY} & 15 & 0 & 66 & \cite{bbm-057} \\
		\hline
		058 & \texttt{ARABIDOPSIS THALIANA CELL~CYCLE} & 14 & 0 & 66 & \cite{bbm-058} \\
		\hline
		059 & \texttt{BORTEZOMIB RESPONSES IN~MYELOMA CELLS} & 62 & 5 & 131 & \cite{bbm-059} \\
		\hline
		060 & \texttt{STOMATAL OPENING} & 44 & 5 & 167 & \cite{bbm-060} \\
		\hline
		061 & \texttt{TUMOR MICROENVIRONMENT IN LYMPHOBLASTIC LEUKAEMIA} & 24 & 2 & 79 & \cite{bbm-061} \\ 
		\hline
		062 & \texttt{CD4 T-CELL DIFFERENTIATION AND PLASTICITY} & 12 & 6 & 78 & \cite{bbm-062} \\ 
		\hline
		063 & \texttt{LAC OPERON} & 10 & 3 & 22 & \cite{bbm-063} \\ 
		\hline
		064 & \texttt{METABOLIC INTERACTIONS IN GUT MICROBIOME} & 8 & 4 & 27 & \cite{bbm-064} \\ 
		\hline
		065 & \texttt{TUMOUR CELL INVASION AND MIGRATION} & 30 & 2 & 156 & \cite{bbm-065-086} \\ 
		\hline
		066 & \texttt{CD4 T-CELL DIFFERENTIATION} & 29 & 9 & 96 & \cite{cell-collective} \\ 
		\hline
		067 & \texttt{REGULATION OF L-ARABINOSE OPERON} & 9 & 4 & 18 & \cite{bbm-067} \\ 
		\hline
		068 & \texttt{AURORA KINASE-A IN NEUROBLASTOMA} & 19 & 4 & 43 & \cite{bbm-068} \\ 
		\hline
		069 & \texttt{IRON ACQUISITION AND STRESS RESPONSE} & 20 & 2 & 38 & \cite{bbm-069} \\ 
		\hline
		070 & \texttt{MAPK CANCER CELL FATE} & 49 & 4 & 104 & \cite{bbm-070-089-090-091} \\ 
		\hline
		071 & \texttt{CASTRATION RESISTANT PROSTATE CANCER} & 28 & 14 & 51 & \cite{bbm-071} \\ 
		\hline
		072 & \texttt{LYMPHOPOIESIS REGULATORY NETWORK} & 67 & 14 & 160 & \cite{bbm-072} \\ 
		\hline
		073 & \texttt{LYMPHOID AND MYELOID CELL SPECIFICATION} & 31 & 2 & 94 & \cite{bbm-073-177} \\ 
		\hline
		074 & \texttt{T-LGL SURVIVAL NETWORK 2011 REDUCED} & 18 & 0 & 43 & \cite{bbm-025-074} \\ 
		\hline
		075 & \texttt{INFLAMMATORY BOWEL DISEASE} & 47 & 0 & 287 & \cite{bbm-075} \\ 
		\hline
		076 & \texttt{SENESCENCE ASSOCIATED SECRETORY PHENOTYPE} & 49 & 2 & 96 & \cite{bbm-076} \\ 
		\hline
		077 & \texttt{SIGNALLING PATHWAY FOR BUTANOL PRODUCTION} & 53 & 13 & 139 & \cite{bbm-077} \\ 
		\hline
		078 & \texttt{IMMUNE SYSTEM} & 151 & 13 & 506 & \cite{cell-collective} \\ 
		\hline
		079 & \texttt{COLORECTAL TUMORIGENESIS} & 184 & 13 & 747 & \cite{bbm-079} \\ 
		\hline
		080 & \texttt{TCR SIGNALING 2018} & 95 & 15 & 212 & \cite{bbm-080-081-082} \\ 
		\hline
		081 & \texttt{TLR5 SIGNALING 2018} & 40 & 2 & 68 & \cite{bbm-080-081-082} \\
		\hline
		082 & \texttt{TCR-TLR5 SIGNALING 2018} & 112 & 16 & 257 & \cite{bbm-080-081-082} \\
		\hline
		083 & \texttt{SIGNALING IN PROSTATE CANCER} & 122 & 11 & 420 & \cite{bbm-083} \\
		\hline
	\end{tabular}

	\begin{tabular}{ | C{16pt} | C{149pt} | C{25pt} | C{25pt} | C{25pt} | C{36pt} | }
		\hline
		ID & Name & Vars. & Inps. & Regs. & Source \\ 
		\hline
		084 & \texttt{ABA INDUCED STOMATAL CLOSURE} & 58 & 23 & 155 & \cite{bbm-084} \\
		\hline
		085 & \texttt{REPROGRAMMING TESTES DERIVED STEM CELLS} & 23 & 4 & 49 & \cite{bbm-085} \\
		\hline
		086 & \texttt{TUMOUR CELL INVASION AND MIGRATION REDUCED} & 18 & 2 & 88 & \cite{bbm-065-086} \\
		\hline
		087 & \texttt{INFLAMMATORY GENE EXPRESSION IN MACROPHAGES} & 99 & 34 & 190 & \cite{bbm-087} \\
		\hline
		088 & \texttt{MIR-9 NEUROGENESIS} & 6 & 0 & 11 & \cite{bbm-088} \\
		\hline
		089 & \texttt{MAPK REDUCED 1} & 13 & 4 & 78 & \cite{bbm-070-089-090-091} \\
		\hline
		090 & \texttt{MAPK REDUCED 2} & 14 & 4 & 60 & \cite{bbm-070-089-090-091} \\
		\hline
		091 & \texttt{MAPK REDUCED 3} & 12 & 4 & 58 & \cite{bbm-070-089-090-091} \\
		\hline
		092 & \texttt{HEPATOCELLULAR CARCINOMA} & 61 & 8 & 139 & \cite{bbm-092-174} \\
		\hline
		093 & \texttt{IMMUNE CHECKPOINT INHIBITORS} & 51 & 15 & 128 & \cite{bbm-093} \\
		\hline
		094 & \texttt{MACROPHAGE POLARIZATION STATES} & 23 & 8 & 52 & \cite{bbm-094} \\
		\hline
		095 & \texttt{FISSION YEAST 2008} & 9 & 1 & 27 & \cite{bbm-095} \\
		\hline
		096 & \texttt{ERBB REGULATED G1-S TRANSITION} & 19 & 1 & 48 & \cite{bbm-096} \\
		\hline
		097 & \texttt{DROSOPHILA WINGS AP} & 8 & 2 & 14 & \cite{bbm-097} \\
		\hline
		098 & \texttt{MACROPHAGE POLARIZATION EXTENDED} & 30 & 12 & 72 & \cite{bbm-098} \\
		\hline
		099 & \texttt{YEAST HYPHAL TRANSITION} & 16 & 3 & 36 & \cite{bbm-099} \\
		\hline
		100 & \texttt{ACUTE MYELOID LEUKEMIA} & 18 & 3 & 30 & \cite{bbm-100} \\
		\hline
		101 & \texttt{GUARD CELL CO2 SIGNALLING} & 61 & 24 & 155 & \cite{bbm-101} \\
		\hline
		102 & \texttt{PANCREATIC CANCER MICROENVIRONMENT REDUCED} & 17 & 5 & 52 & \cite{bbm-102-103} \\
		\hline
		103 & \texttt{PANCREATIC CANCER MICROENVIRONMENT} & 65 & 4 & 110 & \cite{bbm-102-103} \\
		\hline
		104 & \texttt{DROSOPHILA CELL CYCLE} & 11 & 3 & 42 & \cite{bbm-104} \\
		\hline
		105 & \texttt{PLANT GUARD CELL SIGNALLING} & 46 & 3 & 113 & \cite{bbm-105} \\
		\hline
		106 & \texttt{STOMATAL RESTING STATE} & 58 & 22 & 151 & \cite{bbm-106} \\
		\hline
		107 & \texttt{DNA DAMAGE INDUCED AUTOPHAGY} & 32 & 1 & 99 & \cite{bbm-107} \\
		\hline
		108 & \texttt{GEROCONVERSION} & 23 & 2 & 67 & \cite{bbm-108} \\
		\hline
		109 & \texttt{ASYMMETRIC CELL DIVISION A} & 5 & 0 & 15 & \cite{bbm-109-110} \\
		\hline
		110 & \texttt{ASYMMETRIC CELL DIVISION B} & 9 & 0 & 12 & \cite{bbm-109-110} \\
		\hline
		111 & \texttt{APOPTOSIS} & 18 & 15 & 40 & \cite{bbm-covid-disease-map} \\
		\hline
		112 & \texttt{COAGULATION PATHWAY} & 85 & 27 & 195 & \cite{bbm-covid-disease-map} \\
		\hline
		113 & \texttt{ER STRESS} & 107 & 75 & 266 & \cite{bbm-covid-disease-map} \\
		\hline
		114 & \texttt{ETC} & 46 & 38 & 154 & \cite{bbm-covid-disease-map} \\
		\hline
		115 & \texttt{E PROTEIN} & 17 & 18 & 40 & \cite{bbm-covid-disease-map} \\
		\hline
		116 & \texttt{HMOX1 PATHWAY} & 89 & 55 & 228 & \cite{bbm-covid-disease-map} \\
		\hline
		117 & \texttt{IFN LAMBDA} & 28 & 19 & 52 & \cite{bbm-covid-disease-map} \\
		\hline
		118 & \texttt{INTERFERON 1} & 66 & 55 & 190 & \cite{bbm-covid-disease-map} \\
		\hline
 	\end{tabular}	
 
 	\begin{tabular}{ | C{16pt} | C{149pt} | C{25pt} | C{25pt} | C{25pt} | C{36pt} | }
 		\hline
 		ID & Name & Vars. & Inps. & Regs. & Source \\ 
 		\hline
		119 & \texttt{JNK PATHWAY} & 13 & 6 & 21 & \cite{bbm-covid-disease-map} \\
 		\hline
 		120 & \texttt{KYNURENINE PATHWAY} & 78 & 72 & 304 & \cite{bbm-covid-disease-map} \\
 		\hline
 		121 & \texttt{NLRP3 ACTIVATION} & 39 & 18 & 91 & \cite{bbm-covid-disease-map} \\
 		\hline
		122 & \texttt{NSP14} & 74 & 94 & 558 & \cite{bbm-covid-disease-map} \\
		\hline
		123 & \texttt{NSP4 NSP6} & 43 & 17 & 62 & \cite{bbm-covid-disease-map} \\
		\hline
		124 & \texttt{NSP9 PROTEIN} & 119 & 133 & 257 & \cite{bbm-covid-disease-map} \\
		\hline
		125 & \texttt{ORF10 CUL2 PATHWAY} & 34 & 17 & 92 & \cite{bbm-covid-disease-map} \\
		\hline
		126 & \texttt{ORF3A} & 24 & 18 & 56 & \cite{bbm-covid-disease-map} \\
 		\hline
 		127 & \texttt{PAMP SIGNALING} & 44 & 35 & 109 & \cite{bbm-covid-disease-map} \\
 		\hline
 		128 & \texttt{PYRIMIDINE DEPRIVATION} & 56 & 34 & 131 & \cite{bbm-covid-disease-map} \\
 		\hline
 		129 & \texttt{RTC AND TRANSCRIPTION} & 33 & 1 & 40 & \cite{bbm-covid-disease-map} \\
 		\hline
 		130 & \texttt{RENIN ANGIOTENSIN} & 43 & 34 & 130 & \cite{bbm-covid-disease-map} \\
 		\hline
 		131 & \texttt{TGFB PATHWAY} & 7 & 14 & 24 & \cite{bbm-covid-disease-map} \\
 		\hline
 		132 & \texttt{VIRUS REPLICATION CYCLE} & 129 & 19 & 268 & \cite{bbm-covid-disease-map} \\
 		\hline
 		133 & \texttt{ROOT STEM CELL 2010} & 8 & 2 & 16 & \cite{bbm-133} \\
 		\hline
		134 & \texttt{RHEUMATOID ARTHRITIS} & 35 & 3 & 59 & \cite{bbm-134} \\
 		\hline
 		135 & \texttt{SIGNAL TRANSDUCTION} & 28 & 2 & 33 & \cite{bbm-135} \\
 		\hline
 		136 & \texttt{EGF TNF ALPHA SIGNALLING PATHWAY} & 26 & 2 & 31 & \cite{sbml-qual} \\
 		\hline
 		137 & \texttt{SIGNALLING IN LIVER CANCER} & 71 & 11 & 118 & \cite{bbm-137} \\
 		\hline
 		138 & \texttt{APOPTOSIS-UPDATED} & 17 & 14 & 37 & \cite{bbm-138} \\
 		\hline
 		139 & \texttt{ACUTE RESPONSES DURING HYPERINSULINEMIA} & 10 & 9 & 64 & \cite{bbm-139} \\
 		\hline
		140 & \texttt{BREAST CANCER DRUG RESISTANCE} & 62 & 18 & 211 & \cite{bbm-140} \\
 		\hline
 		141 & \texttt{HIGH OSMOLARITY AND MATING PATHWAYS} & 43 & 2 & 94 & \cite{bbm-141} \\
 		\hline
 		142 & \texttt{BLOOD STEM CELL} & 27 & 2 & 126 & \cite{bbm-142} \\
 		\hline
		143 & \texttt{BREAST CANCER INHIBITORS} & 71 & 26 & 223 & \cite{bbm-143} \\
 		\hline
		144 & \texttt{SNF1-AMPK-PATHWAY} & 146 & 56 & 482 & \cite{bbm-144} \\
 		\hline
 		145 & \texttt{MELANOGENESIS} & 61 & 1 & 113 & \cite{bbm-145} \\
 		\hline
 		146 & \texttt{BUDDING YEAST FAURE 2009} & 40 & 10 & 271 & \cite{bbm-146-147-159-180} \\
 		\hline
 		147 & \texttt{BUDDING YEAST EXIT MODULE} & 11 & 5 & 44 & \cite{bbm-146-147-159-180} \\
 		\hline
 		148 & \texttt{AGS CELL FATE DECISION} & 83 & 0 & 185 & \cite{bbm-148-149} \\
 		\hline
 		149 & \texttt{AGS CELL FATE DECISION REDUCED} & 12 & 2 & 62 & \cite{bbm-148-149} \\
 		\hline
 		150 & \texttt{CELL FATE DECISION MULTISCALE} & 31 & 2 & 52 & \cite{bbm-008-150} \\
 		\hline
 		151 & \texttt{TCR REDOX METABOLISM} & 130 & 3 & 417 & \cite{bbm-151-152} \\
 		\hline
 		152 & \texttt{TCR REDOX METABOLISM REDUCED} & 50 & 3 & 288 & \cite{bbm-151-152} \\
 		\hline
 		153 & \texttt{CONTROL OF PROLIFERATION} & 17 & 1 & 34 & \cite{bbm-153} \\
 		\hline
 		154 & \texttt{CONTROL OF TH1 TH2 DIFFERENTIATION} & 18 & 3 & 50 & \cite{bbm-154} \\
 		\hline
 	\end{tabular}
 
	 \begin{tabular}{ | C{16pt} | C{149pt} | C{25pt} | C{25pt} | C{25pt} | C{36pt} | }
	 	\hline
	 	ID & Name & Vars. & Inps. & Regs. & Source \\ 	 	
	 	\hline
	 	155 & \texttt{CONTROL OF TH1 TH2 TH17 TREG DIFFERENTATION} & 45 & 26 & 164 & \cite{bbm-155-156} \\
	 	\hline
	 	156 & \texttt{CONTROL OF TH1 TH2 TH17 TREG DIFFERENTATION REDUCED} & 23 & 13 & 108 & \cite{bbm-155-156} \\
	 	\hline
	 	157 & \texttt{CONTROL OF TH DIFFERENTATION} & 62 & 41 & 234 & \cite{bbm-157} \\
	 	\hline
	 	158 & \texttt{LAMBDA PHAGE LYSOGENY} & 7 & 0 & 30 & \cite{bbm-158} \\
	 	\hline
	 	159 & \texttt{BUDDING YEAST CORE} & 31 & 8 & 158 & \cite{bbm-146-147-159-180} \\
	 	\hline
	 	160 & \texttt{IL17 DIFFERENTIAL EXPRESSION} & 76 & 16 & 246 & \cite{bbm-160} \\
	 	\hline
	 	161 & \texttt{DIFFERENTIATION OF MONOCYTES} & 94 & 2 & 244 & \cite{bbm-161} \\
	 	\hline
	 	162 & \texttt{DROSOPHILA DPP PATHWAY} & 10 & 8 & 40 & \cite{bbm-drosophila} \\
	 	\hline
	 	163 & \texttt{DROSOPHILA EGF PATHWAY} & 24 & 10 & 84 & \cite{bbm-drosophila} \\
	 	\hline
	 	164 & \texttt{EGGSHELL PATTERNING MECHANISTIC} & 17 & 7 & 62 & \cite{bbm-164-165} \\
	 	\hline
	 	165 & \texttt{EGGSHELL PATTERNING PHENOMOENOLOGICAL} & 4 & 4 & 16 & \cite{bbm-164-165} \\
	 	\hline
	 	166 & \texttt{DROSOPHILA JAK STAT PATHWAY} & 7 & 12 & 36 & \cite{bbm-drosophila} \\
	 	\hline
	 	167 & \texttt{MESODERM SPECIFICATION IN DROSOPHILA} & 41 & 16 & 130 & \cite{bbm-167} \\
	 	\hline
	 	168 & \texttt{DROSOPHILA NOTCH PATHWAY} & 7 & 6 & 26 & \cite{bbm-drosophila} \\
	 	\hline
	 	169 & \texttt{DROSOPHILA GAP A} & 5 & 2 & 17 & \cite{bbm-169-170-171-172} \\
	 	\hline
	 	170 & \texttt{DROSOPHILA GAP B} & 4 & 3 & 15 & \cite{bbm-169-170-171-172} \\
	 	\hline
	 	171 & \texttt{DROSOPHILA GAP C} & 5 & 2 & 20 & \cite{bbm-169-170-171-172} \\
	 	\hline
	 	172 & \texttt{DROSOPHILA GAP D} & 5 & 2 & 12 & \cite{bbm-169-170-171-172} \\
	 	\hline
	 	173 & \texttt{GLUCOSE REPRESSION SIGNALLING} & 62 & 14 & 116 & \cite{bbm-048-173} \\
	 	\hline
	 	174 & \texttt{HEPATOCELLULAR CARCINOMA REDUCED} & 19 & 0 & 71 & \cite{bbm-092-174} \\
	 	\hline
	 	175 & \texttt{SEA URCHIN} & 32 & 9 & 95 & \cite{bbm-175} \\
	 	\hline
	 	176 & \texttt{MYELOFIBROTIC MICROENVIRONMENT} & 47 & 2 & 148 & \cite{bbm-176} \\
	 	\hline
	 	177 & \texttt{LYMPHOID CELL SPECIFICATION} & 31 & 3 & 96 & \cite{bbm-073-177} \\
	 	\hline
	 	178 & \texttt{MAST CELL ACTIVATION} & 45 & 3 & 72 & \cite{bbm-178} \\
	 	\hline
	 	179 & \texttt{MICROENVIRONMENT CONTROL} & 46 & 10 & 149 & \cite{bbm-179} \\
	 	\hline
	 	180 & \texttt{MORPHOGENETIC CHECKPOINT} & 11 & 1 & 36 & \cite{bbm-146-147-159-180} \\
	 	\hline
	 	181 & \texttt{MULTILEVEL CELL CYCLE} & 12 & 1 & 59 & \cite{bbm-181-182} \\
	 	\hline
	 	182 & \texttt{BOOLEAN CELL CYCLE} & 12 & 2 & 59 & \cite{bbm-181-182} \\
	 	\hline
	 	183 & \texttt{ALTERATIONS IN BLADDER} & 31 & 4 & 111 & \cite{bbm-183} \\
	 	\hline
	 	184 & \texttt{P53 MDM2 NETWORK} & 5 & 1 & 15 & \cite{bbm-184} \\
	 	\hline
	 	185 & \texttt{CHICKEN SEX DETERMINATION} & 12 & 3 & 34 & \cite{bbm-185-186} \\
	 	\hline	 	
	\end{tabular}

	\begin{tabular}{ | C{16pt} | C{149pt} | C{25pt} | C{25pt} | C{25pt} | C{36pt} | }
		\hline
		ID & Name & Vars. & Inps. & Regs. & Source \\ 	 	
		\hline
		186 & \texttt{CHICKEN SEX DETERMINATION REDUCED} & 7 & 3 & 25 & \cite{bbm-185-186} \\
		\hline
		187 & \texttt{MAMMAL SEX DETERMINATION 1 CELL} & 13 & 6 & 55 & \cite{bbm-187-188} \\
		\hline
		188 & \texttt{MAMMAL SEX DETERMINATION 2 CELL} & 30 & 7 & 132 & \cite{bbm-187-188} \\
		\hline
		189 & \texttt{TRP BIOSYNTHESIS} & 5 & 1 & 13 & \cite{bbm-189} \\
		\hline
		190 & \texttt{BRAF TREATMENT RESPONSE} & 32 & 5 & 74 & \cite{bbm-190} \\
		\hline
		191 & \texttt{SEGMENT POLARITY 1 CELL} & 17 & 2 & 55 & \cite{bbm-191-192} \\
		\hline
		192 & \texttt{SEGMENT POLARITY 6 CELL} & 102 & 0 & 352 & \cite{bbm-191-192} \\
		\hline
		193 & \texttt{SENESCENCE G1S CHECKPOINT} & 28 & 2 & 100 & \cite{bbm-193} \\
		\hline
	 	194 & \texttt{VULVAR PRECURSOR CELLS} & 78 & 28 & 236 & \cite{bbm-194} \\
		\hline
	 	195 & \texttt{CTLA4 PD1 CHECKPOINT INHIBITORS} & 161 & 55 & 439 & \cite{bbm-195} \\
		\hline
		196 & \texttt{T-LYMPHOCYTE SPECIFICATION} & 58 & 3 & 237 & \cite{bbm-196} \\
		\hline
		197 & \texttt{ANTERIOR POSTERIOR BOUNDARY} & 48 & 8 & 253 & \cite{bbm-197} \\
		\hline
		198 & \texttt{PAIR RULE MODULE} & 11 & 0 & 48 & \cite{bbm-198} \\
		\hline
		199 & \texttt{HEPATOCELLULAR CARCINOMA COMPARTMENTALIZED} & 30 & 0 & 97 & \cite{bbm-199} \\
		\hline	 	
		200 & \texttt{LUNG-CANCER-CELL-CYCLE} & 25 & 1 & 71 & \cite{bbm-200} \\
		\hline	 			
		201 & \texttt{ONKOGENE ROLE OF INCRNA ANRIL} & 31 & 2 & 94 & \cite{bbm-201} \\
		\hline
		202 & \texttt{ONKOGENE ROLE OF INCRNA XIST} & 30 & 1 & 107 & \cite{bbm-202} \\
		\hline
		203 & \texttt{DDR SIGNALLING PATHWAYS} & 39 & 1 & 143 & \cite{bbm-203} \\
		\hline
		204 & \texttt{HUMAN BRAIN ORGANOIDS} & 76 & 1 & 205 & \cite{bbm-204} \\
		\hline
		205 & \texttt{EPITHELIAL-MESENCHYMAL TRANSITION IN BLADDER} & 41 & 4 & 118 & \cite{bbm-205-206} \\
		\hline
		206 & \texttt{EPITHELIAL-MESENCHYMAL TRANSITION IN BREAST} & 36 & 5 & 97 & \cite{bbm-205-206} \\
		\hline
		207 & \texttt{BREAST CANCER TUMOUR} & 85 & 18 & 441 & \cite{bbm-207} \\
		\hline
		208 & \texttt{HEMATOPOIESIS AGING} & 15 & 0 & 36 & \cite{bbm-208} \\
		\hline
		209 & \texttt{ABERRANT CELL CYCLE PROGRESSION} & 85 & 2 & 372 & \cite{bbm-209} \\
		\hline
		210 & \texttt{DRUG SYNERGY PREDICTION} & 144 & 0 & 367 & \cite{bbm-210} \\
		\hline
		211 & \texttt{EPITHELIAL-DERIVED CANCER CELLS} & 183 & 0 & 602 & \cite{bbm-211} \\
		\hline
		212 & \texttt{ESCHERICHIA COLI TRYPTOPHAN} & 13 & 3 & 27 & \cite{bbm-212} \\
		\hline
%		213 & \texttt{----} & -- & -- & -- & -- \\
%		\hline
%		214 & \texttt{----} & -- & -- & -- & -- \\
%		\hline
%		215 & \texttt{----} & -- & -- & -- & -- \\
%		\hline
%		216 & \texttt{----} & -- & -- & -- & -- \\
%		\hline
%		217 & \texttt{----} & -- & -- & -- & -- \\
%		\hline
%		218 & \texttt{----} & -- & -- & -- & -- \\
%		\hline
%		219 & \texttt{----} & -- & -- & -- & -- \\
%		\hline
%		220 & \texttt{----} & -- & -- & -- & -- \\
%		\hline
%		221 & \texttt{----} & -- & -- & -- & -- \\
%		\hline
	\end{tabular}

%	\begin{tabular}{ | C{16pt} | C{149pt} | C{25pt} | C{25pt} | C{25pt} | C{36pt} | }
%		\hline
%		ID & Name & Vars. & Inps. & Regs. & Source \\ 	 	
%		\hline
%	\end{tabular}
\end{center}

\bibliographystyle{plain}
\bibliography{report}

\end{document}